\documentclass[aspectratio=169]{beamer}

% Theme selection
\usetheme{Madrid}
\useoutertheme{miniframes} % Adds top navigation with circles
\usecolortheme{beaver}

% Packages for graphics and text encoding
\usepackage{graphicx}
\usepackage[utf8]{inputenc}
\usepackage[spanish]{babel}
\usepackage{caption}

% Custom Footer Configuration (Infolines style)
% Madrid theme already provides a good footer, but we'll enforce the specific format requested
\setbeamertemplate{navigation symbols}{} % Remove default navigation symbols

% Title information
\title[Avances Proyecto D. 72044]{Generación de Energía Eólica en Plataformas Marinas en Desuso}
\subtitle{Reporte de Actividades: Fase 1}
\author[F. Cisneros]{M. I. Francisco Cisneros}
\institute[IMP]{Instituto Mexicano del Petróleo}
\date{19 de Febrero de 2026}

\begin{document}

% Title Page Customization
\begin{frame}[plain]
    \centering
    \vspace{0.5cm}
    \includegraphics[width=4cm]{../000_logo.png}
    
    \vspace{0.5cm}
    
    {\Large \textbf{Generación de Energía Eólica en Plataformas Marinas en Desuso}}
    \vspace{0.3cm}
    
    {\large Reporte de Actividades: Fase 1}
    \vspace{1.0cm}
    
    {\textbf{M. I. Francisco Cisneros}}
    \vspace{0.2cm}
    
    {Doctorado en Ingeniería}
    \vspace{0.2cm}
    
    {\textbf{Instituto Mexicano del Petróleo}}
    \vspace{1.0cm}
    
    {19 de Febrero de 2026}
\end{frame}

% Table of Contents
\begin{frame}{Contenido}
    \tableofcontents
\end{frame}

% -----------------------------------------------------------------------------
% SECCIÓN 1: ACTIVIDAD 1
% -----------------------------------------------------------------------------
\section{Actividad 1: Modelado Estructural Preliminar en SACS}

% Frame 1: Vista General Alámbrica
\begin{frame}{Configuración Geométrica Global}
    \begin{figure}
        \centering
        \includegraphics[width=0.7\textwidth, height=0.6\textheight, keepaspectratio]{../primera_actividad/figs_primera_actividad/001_vista_general_alambrica_de_la_plataforma.png}
        \caption{Vista general alámbrica de la plataforma}
    \end{figure}
    \begin{itemize}
        \item Definición de la geometría global mediante estructura alámbrica.
        \item Configuración inicial de elementos principales y secundarios basado en planos estructurales.
    \end{itemize}
\end{frame}

% Frame 2: Conectividad en Lecho Marino
\begin{frame}{Definición de Cimentación}
    \begin{figure}
        \centering
        \includegraphics[width=0.7\textwidth, height=0.6\textheight, keepaspectratio]{../primera_actividad/figs_primera_actividad/002_conectividad_y_triangulacion_en_el_lecho_marino.png}
        \caption{Conectividad y triangulación en el lecho marino}
    \end{figure}
    \begin{itemize}
        \item Modelado de la cimentación incluyendo la interacción con el suelo marino.
        \item Establecimiento de la conectividad y triangulación base para la estabilidad.
    \end{itemize}
\end{frame}

% Frame 3: Verificación Nodal
\begin{frame}{Validación de Conectividad}
    \begin{figure}
        \centering
        \includegraphics[width=0.7\textwidth, height=0.6\textheight, keepaspectratio]{../primera_actividad/figs_primera_actividad/003_verificacion_de_conectividad_nodal_en_base.png}
        \caption{Verificación de conectividad nodal en base}
    \end{figure}
    \begin{itemize}
        \item Revisión de nudos y condiciones de soporte en la base de la estructura.
        \item Verificación básica para asegurar la consistencia del modelo numérico.
    \end{itemize}
\end{frame}

% Frame 4: Detalle Sólido Piernas/Arriostres
\begin{frame}{Detalle de Uniones}
    \begin{figure}
        \centering
        \includegraphics[width=0.7\textwidth, height=0.6\textheight, keepaspectratio]{../primera_actividad/figs_primera_actividad/004_detalle_en_vista_solida_de_encuentros_piernas_arriostres.png}
        \caption{Detalle en vista sólida de encuentros piernas-arriostres}
    \end{figure}
    \begin{itemize}
        \item Modelado detallado de la intersección entre piernas y elementos de arriostramiento.
        \item Inspección visual en modo sólido para detectar interferencias o desconexiones.
    \end{itemize}
\end{frame}

% Frame 5: Grupo de Miembros
\begin{frame}{Administración de Propiedades}
    \begin{figure}
        \centering
        \includegraphics[width=0.7\textwidth, height=0.6\textheight, keepaspectratio]{../primera_actividad/figs_primera_actividad/005_ventana_de_grupo_de_miembros_con_secc04.png}
        \caption{Ventana de grupo de miembros con SECC04}
    \end{figure}
    \begin{itemize}
        \item Organización del modelo mediante grupos de miembros para asignación eficiente de propiedades.
        \item Uso de herramientas de edición masiva para estandarizar secciones.
    \end{itemize}
\end{frame}

% Frame 6: Editor de Sección
\begin{frame}{Definición de Secciones Tubulares}
    \begin{figure}
        \centering
        \includegraphics[width=0.7\textwidth, height=0.6\textheight, keepaspectratio]{../primera_actividad/figs_primera_actividad/006_editor_de_seccion_tubular_secc04.png}
        \caption{Editor de sección tubular SECC04}
    \end{figure}
    \begin{itemize}
        \item Configuración de propiedades geométricas y de material para elementos tubulares.
        \item Asignación de parámetros específicos (diámetro, espesor) según especificaciones.
    \end{itemize}
\end{frame}

% Frame 7: Vista Coloreada por Grupos
\begin{frame}{Visualización por Grupos}
    \begin{figure}
        \centering
        \includegraphics[width=0.7\textwidth, height=0.6\textheight, keepaspectratio]{../primera_actividad/figs_primera_actividad/007_vista_solida_coloreada_por_grupos.png}
        \caption{Vista sólida coloreada por grupos}
    \end{figure}
    \begin{itemize}
        \item Verificación visual de la asignación de propiedades mediante código de colores.
        \item Aseguramiento de la uniformidad en la configuración por bahías y niveles.
    \end{itemize}
\end{frame}

% Frame 8: Asignación de Carga Viva
\begin{frame}{Asignación de Cargas}
    \begin{figure}
        \centering
        \includegraphics[width=0.7\textwidth, height=0.6\textheight, keepaspectratio]{../primera_actividad/figs_primera_actividad/008_asignacion_de_carga_viva_a_nudos_de_cubiertas.png}
        \caption{Asignación de carga viva a nudos de cubiertas}
    \end{figure}
    \begin{itemize}
        \item Aplicación de cargas vivas sobre los nodos de las cubiertas principales.
        \item Modelado de solicitaciones para análisis preliminares de la estructura.
    \end{itemize}
\end{frame}

% Frame 9: Comparativa ETABS vs SACS
\begin{frame}{Validación Cruzada}
    \begin{figure}
        \centering
        \includegraphics[width=0.7\textwidth, height=0.6\textheight, keepaspectratio]{../primera_actividad/figs_primera_actividad/009_comparativa_de_geometria_etabs_izquierda_sacs_derecha.png}
        \caption{Comparativa de geometría: ETABS (Izq) vs SACS (Der)}
    \end{figure}
    \begin{itemize}
        \item Comparación geométrica con modelos de referencia (ETABS) para validar la topología.
        \item Consistencia en la geometría global, nodos y elementos entre ambas plataformas.
    \end{itemize}
\end{frame}

% -----------------------------------------------------------------------------
% SECCIÓN 2: ACTIVIDAD 2
% -----------------------------------------------------------------------------
\section{Actividad 2: Desarrollo del Procesador de Fatiga SACS v1.0}

% Frame 1: Visión General
\begin{frame}{Visión General del Proyecto}
    \begin{block}{Objetivo General}
        Desarrollar una aplicación de escritorio para la consolidación automatizada de reportes de fatiga generados por SACS, permitiendo la suma aritmética de daño acumulado entre etapas temporales.
    \end{block}
    
    \vspace{0.5cm}
    
    \begin{itemize}
        \item \textbf{Entregable:} Aplicación standalone con interfaz gráfica (GUI) dedicada.
        \item \textbf{Funcionalidades Clave:}
        \begin{itemize}
            \item Carga y procesamiento masivo de archivos de reporte FTG (\texttt{.txt}).
            \item Normalización de datos y corrección de notación científica.
            \item Suma de daños por elemento estructural único (JOINT + MEMBER + GRUP).
            \item Exportación automatizada de resultados consolidados a Excel.
        \end{itemize}
    \end{itemize}
\end{frame}

% Frame 2: Trabajo Realizado
\begin{frame}{Avance Actual: Etapas de Procesamiento}
    Actualmente se han completado las fases críticas de backend para la ingestión de datos:
    
    \begin{block}{Etapa 1: Limpieza y Normalización}
         Implementación de módulos para transformar datos crudos:
        \begin{itemize}
            \item Algoritmos de corrección para notación científica Fortran (ej. \texttt{.123-4} $\to$ \texttt{1.23E-05}).
            \item Filtrado inteligente de encabezados y líneas no relevantes.
        \end{itemize}
    \end{block}
    
    \begin{block}{Etapa 2: Parsing y Extracción Estructurada}
        Desarrollo de un parser robusto para interpretar reportes SACS:
        \begin{itemize}
            \item Identificación precisa de secciones \textit{MEMBER FATIGUE DETAIL REPORT}.
            \item Extracción de identificadores únicos y casos de carga asociados.
        \end{itemize}
    \end{block}
\end{frame}

% Frame 3: Detalles Técnicos de Extracción
\begin{frame}{Detalles Técnicos: Implementación del Parser}
    El núcleo de extracción utiliza una arquitectura basada en \textbf{Máquina de Estados Finitos} para procesar secuencialmente archivos de texto no estructurados.
    
    \begin{columns}
        \begin{column}{0.48\textwidth}
            \begin{itemize}
                \item \textbf{Estados del Parser:}
                \begin{enumerate}
                    \item \texttt{SEARCHING}: Búsqueda de secciones relevantes.
                    \item \texttt{READING\_HEADER}: Procesamiento de metadatos.
                    \item \texttt{READING\_CASES}: Captura de 16 casos de carga.
                    \item \texttt{READING\_TOTAL}: Extracción acumulada.
                \end{enumerate}
            \end{itemize}
        \end{column}
        \begin{column}{0.48\textwidth}
            \begin{block}{Extracción de Daño}
                Captura de vector de daños circunferenciales (8 puntos):
                \begin{itemize}
                    \item \small \texttt{[TOP, TOP-LEFT, ... , TOP-RIGHT]}
                    \item Identificación de ubicación crítica y daño máximo por elemento.
                \end{itemize}
            \end{block}
        \end{column}
    \end{columns}
\end{frame}

% Frame 4: Mockup GUI
\begin{frame}[fragile]{Prototipo de Interfaz Gráfica (GUI)}
    Diseño conceptual para la Etapa 4: Interfaz standalone para consolidación y exportación.
    
    \begin{center}
    \tiny
    \begin{verbatim}
+--------------------------------------------------------------+
| Procesador de Fatiga SACS v1.0                     [_][ ][X] |
+--------------------------------------------------------------+
| [+] Seleccionar Archivos SACS                                |
|                                                              |
| Archivos seleccionados: (3)                                  |
| +------------------------------------------------------+     |
| | * ftglstE1.txt                          [X]          |     |
| | * ftglstE2.txt                          [X]          |     |
| | * ftglstE3.txt                          [X]          |     |
| +------------------------------------------------------+     |
|                                                              |
| Opciones de exportación:                                     |
| Ruta salida: [C:/resultados/fatiga_consolidada.xlsx] [DIR]   |
| [x] Incluir estadísticas     [x] Incluir gráficos            |
|                                                              |
| Progreso:                                                    |
| [================--------] 65% (Procesando archivo 2/3)      |
|                                                              |
|              [PROCESAR]    [RESULTADOS]    [CERRAR]          |
+--------------------------------------------------------------+
    \end{verbatim}
    \end{center}
\end{frame}

% Frame 5: Trabajo Futuro (Roadmap)
\begin{frame}{Roadmap de Desarrollo: Fases Restantes}
    El plan de desarrollo contempla las siguientes etapas para completar la versión 1.0:
    
    \begin{enumerate}
        \setcounter{enumi}{2} % Comienza en 3
        \item \textbf{Consolidación y Suma (Etapa 3):} 
        \begin{itemize}
            \item Lógica de agregación para múltiples archivos y validación de consistencia.
        \end{itemize}
        
        \item \textbf{Interfaz Gráfica (Etapa 4):} 
        \begin{itemize}
            \item Desarrollo de GUI con \texttt{Tkinter} (selectores, barras de progreso).
        \end{itemize}
        
        \item \textbf{Exportación y Reportes (Etapa 5):} 
        \begin{itemize}
            \item Generación de reportes Excel con formato condicional y gráficos.
        \end{itemize}
        
        \item \textbf{Validación y Despliegue (Etapas 6-7):} 
        \begin{itemize}
            \item Testing exhaustivo, empaquetado (.exe) y documentación final.
        \end{itemize}
    \end{enumerate}
\end{frame}

\end{document}
