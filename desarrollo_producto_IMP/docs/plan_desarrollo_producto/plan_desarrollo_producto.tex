\documentclass{article}
\usepackage[utf8]{inputenc}
\usepackage[spanish]{babel}
\usepackage[T1]{fontenc}
\usepackage{geometry}
\usepackage{xcolor}
\usepackage{listings}
\usepackage{hyperref}
\usepackage{float}

% Configuración de márgenes
\geometry{a4paper, margin=2.5cm}

% Configuración para mostrar código/pseudocódigo
\lstset{
    basicstyle=\ttfamily\small,
    breaklines=true,
    frame=single,
    backgroundcolor=\color{gray!10},
    keywordstyle=\color{blue}\bfseries,
    commentstyle=\color{green!50!black},
    stringstyle=\color{red},
    numbers=left,
    numberstyle=\tiny,
    inputencoding=utf8,
    extendedchars=true,
    literate={á}{{\'a}}1 {é}{{\'e}}1 {í}{{\'i}}1 {ó}{{\'o}}1 {ú}{{\'u}}1 {ñ}{{\~n}}1
}

\title{Documentación Técnica: Automatización de Reportes SACS}
\author{Equipo de Ingeniería de Datos}
\date{\today}

\begin{document}

\maketitle

\section{Resumen Ejecutivo}

El propósito de esta herramienta es \textbf{automatizar la consolidación de reportes de fatiga} generados por el software SACS. El objetivo final es transformar miles de líneas de texto desestructurado en una \textbf{Tabla Maestra de Excel} limpia, donde el daño de fatiga de cada elemento estructural esté sumado correctamente a través de todas las etapas de la vida útil de la plataforma.

\section{El Flujo del Proceso (Lógica Interna)}

El script actúa como un motor de procesamiento de datos ETL (Extract, Transform, Load). El sistema ejecuta el siguiente procedimiento lógico:

\subsection{Paso 1: Localización y Filtrado (El ``Scanner'')}
El archivo de texto contiene muchas secciones irrelevantes (\textit{Input Echo}, etc.).
\begin{itemize}
    \item \textbf{Acción:} El motor de lectura escanea el archivo pero \textbf{ignora todo} hasta encontrar la señal de inicio: \texttt{MEMBER FATIGUE REPORT}.
    \item \textbf{Por qué:} Optimiza la memoria y evita datos falsos positivos.
\end{itemize}

\subsection{Paso 2: Limpieza y Normalización}
\begin{itemize}
    \item \textbf{Filtro de Ruido:} Se eliminan automáticamente los encabezados de página y líneas decorativas de SACS.
    \item \textbf{Reconstrucción de Filas:} Se maneja la lógica para unir registros que el reporte original divide en múltiples líneas de texto.
\end{itemize}

\subsection{Paso 3: Traducción Numérica (El ``Decodificador'')}
SACS utiliza notación científica de formato antiguo (Fortran).
\begin{itemize}
    \item \textbf{Problema:} El valor \texttt{.123-4} no es reconocido por sistemas modernos (debería ser \texttt{1.23E-05}).
    \item \textbf{Solución:} Se aplica una función de Regex \texttt{(?<=\textbackslash d)(?=[+-])} para inyectar la notación "E" y permitir el cálculo matemático.
\end{itemize}

\subsection{Paso 4: Agregación Aritmética (La ``Suma'')}
El núcleo de la herramienta cruza información entre $N$ archivos seleccionados por el usuario.
\begin{itemize}
    \item Si el Archivo A (años 0-10) indica daño \texttt{0.5} en la Junta J101.
    \item Y el Archivo B (años 10-20) indica daño \texttt{0.3} en la misma Junta.
    \item \textbf{Resultado:} El sistema consolida la identidad única y calcula \texttt{0.8} total.
\end{itemize}

\section{Arquitectura e Interfaz de Usuario}

Para facilitar su adopción por parte del equipo de ingeniería civil, la solución se entrega como una **aplicación de escritorio ejecutable** (.exe) con una arquitectura de separación de responsabilidades:

\begin{itemize}
    \item \textbf{Backend:} Lógica de negocio (Python/Pandas) que maneja el parsiing y la matemática.
    \item \textbf{Frontend:} Interfaz gráfica (Tkinter) que abstrae la complejidad del código.
\end{itemize}

\subsection{Diseño de Interacción (Wireframe)}

Se ha diseñado una interfaz minimalista centrada en la eficiencia del flujo de trabajo del ingeniero. A continuación, se presenta el esquema visual de la aplicación propuesta:

\begin{figure}[H]
\centering
\begin{lstlisting}[basicstyle=\ttfamily\small, frame=none, numbers=none, backgroundcolor=\color{white}]
+-------------------------------------------------------+
|  Procesador de Fatiga SACS v1.0             [-][x]    |
+-------------------------------------------------------+
|                                                       |
|   1. SELECCION DE DATOS:                              |
|   [ > Seleccionar archivos .TXT ]                     |
|   > Estado: Se han seleccionado 5 archivos.           |
|                                                       |
|   2. EJECUCION:                                       |
|   [ @ PROCESAR Y GENERAR EXCEL ]                      |
|                                                       |
|   3. MONITOR DE PROGRESO:                             |
|   [|||||||||||||||||||| 100% |||||||||||||||||||||]   |
|   > !Listo! Reporte guardado como 'Consolidado.xlsx'  |
|                                                       |
+-------------------------------------------------------+
\end{lstlisting}
\caption{Wireframe de la Interfaz de Usuario (GUI)}
\label{fig:wireframe}
\end{figure}

Esta interfaz permite al usuario operar la herramienta sin conocimientos previos de programación, facilitando su despliegue en cualquier estación de trabajo Windows.

\section{Lógica del Algoritmo (Pseudocódigo del Backend)}

El siguiente bloque detalla la lógica interna que ejecuta el procesamiento.

\begin{lstlisting}[caption={Pseudocódigo Estructural para Procesamiento SACS}]
ALGORITMO Procesamiento_Fatiga_SACS

DEFINIR PATRONES REGEX:
    inicio_tabla -> "MEMBER FATIGUE REPORT"
    formato_numero_sacs -> (?<=\d)(?=[+-])  (detectar signo sin 'E')
    linea_datos -> Capturar (Junta, Miembro, Grupo, Daño)

AL RECIBIR (lista_archivos) DESDE INTERFAZ:
    CREAR lista vacia "base_de_datos_total"
    
    PARA CADA archivo EN lista_archivos:
        ACTUALIZAR BARRA_PROGRESO_GUI()
        
        PARA CADA linea EN archivo:
            SI linea contiene "ENCABEZADO": SALTAR
            SI linea contiene "MEMBER FATIGUE REPORT": ACTIVAR_CAPTURA
            
            SI ACTIVAR_CAPTURA:
                EXTRAER datos con REGEX
                NORMALIZAR numero (.123-4 -> 1.23E-4)
                AGREGAR a lista_temporal
        
        CONCATENAR lista_temporal A "base_de_datos_total"

    PROCESAR "base_de_datos_total":
        AGRUPAR POR (Junta, Elemento, Grupo)
        SUMAR columna (Daño)
        ORDENAR descendente

    RETORNAR archivo "Reporte_Final.xlsx"
FIN PROGRAMA
\end{lstlisting}

\end{document}